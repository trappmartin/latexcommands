\documentclass[11pt]{article}

\usepackage{graphicx}
\usepackage{verbatim}

\usepackage{pctex}

\title{Examples for pctex}
\author{ Martin Trapp }
\date{}

\begin{document}
\maketitle

\noindent
`pctex` provides some useful commands for working with probabilistic circuits. The main purpose of this is reusability and harmonization of notation.

\section{General/Misc}
\begin{itemize}
	\item Log-sum-exp $\lse{i=1}{k}$: \verb!$\lse{i=1}{k}$!
	\item $\poly{N}$: \verb!$\poly{N}$!
	\item Independent RVs $X_1 \indepsym X_2, \indep{X_1}{X_2}$: \verb!$X_1 \indepsym X_2, \indep{X_1}{X_2}$!
	\item Cond. independent RVs $(\cindep{X_1}{X_2}{X_3})$: \verb!$(\cindep{X_1}{X_2}{X_3})$!
\end{itemize}

\section{General graphs}
\begin{itemize}
	\item Graph $\graph$: \verb!$\graph$!
	\item Walk $\walk$: \verb!$\walk$!
	\item Tree $\tree$: \verb!$\tree$!
	\item Vertex set $\vset(\graph)$: \verb!$\vset(\graph)$!
	\item Edge set $\eset(\graph)$: \verb!$\eset(\graph)$!
	\item Node/nodes $\node, \nodes$: \verb!$\node$!
	\item Child/children: $\cnode, \cnodes$: \verb!$\child$!
	\item Children of a node: $\ch{\node}$: \verb!$\ch{\node}$!
	\item Parents of a node: $\pa{\node}$: \verb!$\pa{\node}$!
	\item Neighbours: $\neigh{\node}$: \verb!$\neigh{\node}$!
\end{itemize}

\section{Probabilistic Circuits}
\begin{itemize}
	\item Probabilistic circuit: $\pc$: \verb!$\pc$!
	\item Scope function: $\scopesym, \scope{\node}$: \verb!$\scopesym, \scope{\node}$!
	\item v-tree: $\vtree$: \verb!$\vtree$!
	\item Sum node/nodes: $\snode, \snodes$: \verb!$\snode, \snodes$!
	\item Product node/nodes: $\pnode, \pnodes$: \verb!$\pnode, \pnodes$!
	\item Leaf node/nodes: $\lnode, \lnodes$: \verb!$\lnode, \lnodes$!
	\item Region/regions: $\region, \regions$: \verb!$\region, \regions$!
	\item Partition/partitions: $\partition, \partitions$: \verb!$\partition, \partitions$!
	\item Region-graph: $\rg$: \verb!$\rg$!
\end{itemize}

\section{Tikz / Plotting}
Plotting is based on an adaptation of `tikzlibraryspn.code.tex` by Nicola Di Mauro and Antonio Vergari.

\noindent
Examples to illustrate how to use the plotting:
\begin{figure}[h!]
\centering
\begin{tikzpicture}[
	>=latex,
	level/.style={sibling distance = 2cm/(#1), 
	level distance = 1.2cm},
	edge from parent/.style={draw,-latex}
	]
\node[sum] (s1) {}
    child {node[prod]  (p1) {}
    	child {node[bern, label=below:{$X_1$}]  (s2) {}}
	child {node[cont, label=below:{$X_2$}]  (s3) {}}
    }
    child {node[prod, draw=orange] (p2) {}
        	child {node[cat, label=below:{$X_1$}]  (s2) {}}
	child {node[cont, label=below:{$X_2$}]  (s3) {}}
    };
    
\draw[->] (s1) -- node[left] {$\theta_1$} (p1);
\draw[->] (s1) -- node[right] {$\theta_2$} (p2);
\end{tikzpicture}
\hfill
\begin{tikzpicture}

\node[sum]  at (0,1) {};
\node[prod] at (1,1) {};
\node[max] at (2,1) {};
\node[land] at (3,1) {};
\node[lor] at (4,1) {};

\node[cont] at (0,0) {};
\node[bern] at (1,0) {};
\node[cat] at (2,0) {};
\node[pcnode] (c) at (3,0) {\large$\top$};
\node (t) at (3,-1) {some custom node};

\draw[->] (t) -- (c);

\end{tikzpicture}
\end{figure}

Code for the figures above:
\subsection*{Figure 1}
\begin{verbatim}
\begin{tikzpicture}[
	>=latex,
	level/.style={sibling distance = 2cm/(#1), 
	level distance = 1.2cm},
	edge from parent/.style={draw,-latex}
	]
\node[sum] (s1) {}
    child {node[prod]  (p1) {}
    	child {node[bern, label=below:{$X_1$}]  (s2) {}}
	child {node[cont, label=below:{$X_2$}]  (s3) {}}
    }
    child {node[prod, draw=orange] (p2) {}
        	child {node[cat, label=below:{$X_1$}]  (s2) {}}
	child {node[cont, label=below:{$X_2$}]  (s3) {}}
    };
    
\draw[->] (s1) -- node[left] {$\theta_1$} (p1);
\draw[->] (s1) -- node[right] {$\theta_2$} (p2);
\end{tikzpicture}
\end{verbatim}

\subsection*{Figure 2 / Reference}
\begin{verbatim}
\begin{tikzpicture}

\node[sum]  at (0,1) {};
\node[prod] at (1,1) {};
\node[max] at (2,1) {};
\node[land] at (3,1) {};
\node[lor] at (4,1) {};

\node[cont] at (0,0) {};
\node[bern] at (1,0) {};
\node[cat] at (2,0) {};
\node[pcnode] (c) at (3,0) {\large$\top$};
\node (t) at (3,-1) {some custom node};

\draw[->] (t) -- (c);

\end{tikzpicture}
\end{verbatim}

\end{document}
